%% LyX 1.1 created this file.  For more info, see http://www.lyx.org/.
%% Do not edit unless you really know what you are doing.
\documentclass{article}
\usepackage[T1]{fontenc}
\usepackage[latin1]{inputenc}

\makeatletter


%%%%%%%%%%%%%%%%%%%%%%%%%%%%%% LyX specific LaTeX commands.
\providecommand{\LyX}{L\kern-.1667em\lower.25em\hbox{Y}\kern-.125emX\@}

\makeatother

\begin{document}


\title{Mextools}

\maketitle

\section{Einf�hrung\label{intro}}

Bei den Mextools-Files handelt es sich um eine Sammlung von C++ Header und Programmdateien,
die die Arbeit mit Mex-Dateien, also C bzw. C++ Source-Code, der kompiliert
und dynamisch zu Matlab angelinkt wird und dann dort als Funktion zur Verf�gung
steht. Die Mextools-Files erm�glichen es, bestehenden Mex-Source-Code nahezu
unver�ndert entweder zu einem standalone Unix-Binary oder zu einem Octave-Okt-File
zu kompilieren. 

Dazu mu� im Mex-Source-Code lediglich statt des obligatorischen mex.h die Datei
mextools.h included werden. Wozu der Code dann kompiliert wird, entscheidet
dann allein der Aufruf, mit dem der Code kompiliert wird:

\begin{enumerate}
\item Wird wie gewohnt von der Matlab-Kommandozeile aus mit dem Befehl {\tt mex file.cpp}
kompiliert, entsteht wie gewohnt die Mex-Dll-Datei, die dann wie ein eingebautes
Matlab-Kommando verwendet werden kann.
\item Wird von der Unix-Kommandozeile aus mit dem normalen C/C++ Kompiler (z.B. {\tt g++ file.cpp})
kompiliert, entsteht ein standalone Unix-Binary, das seine Eingabeargumente
aus einer Argumentendatei liest und evtl. einigen ASCII-Datens�tzen liest und
das seine Ausgabeargumente in ASCII-Dateien schreibt.
\item Wird von der Unix-Kommandozeile aus mit dem Befehl {\tt mkoctfile file.cpp}
kompiliert, entsteht ein Octave-Okt-File, der von dem freien Matlab-Clone Octave
aus �hnlich der Matlab-Mex-DLL benutzt werden kann.
\end{enumerate}

\section{Einschr�nkugen}

In Anwendungsart 1 (siehe \ref{intro}) steht nat�rlich wie gewohnt die volle
Funktionalit�t der Mex-Api zur Verf�gung, sie wird ja von Matlab selbst bereitgestellt.
In den beiden anderen Anwendungsarten wird die Mex-Api von dem Mextools-Code
emuliert, daher steht nur eine Auswahl der Mex-Api zur Verf�gung. Momentan sind
nur einfache numerische mxArrays und einfache Character mxArrays implementiert.
Weder Strukturen, Cell-Arrays oder Objekte sind verf�gbar. Auch sind keine Matlab-Engine
oder I/O-Aufrufe m�glich.


\section{Standalone Unix-Binaries}

Die M�glichkeit, aus Mx-Source-Code ein Unix-Binary zu erzeugen, ist besonders
interessant im Hinblick, wenn Source-Code, den man anfangs bequem unter Matlab
entwickelt hat, auf sehr gro�e Probleme angewendet werden soll, wo evtl. Rechenzeiten
von mehreren Tagen entstehen k�nnen. W�hrend dieser Zeit sollen aber sicher
die teuren Lizenzen nicht unn�tig blockiert sein.

Nach dem Kompilieren kann das Binary (nennen wir es a.out) mit folgendem Syntax
von der Unix-Kommandozeile aus aufgerufen werden: {\code a.out argumentfile.txt nr_arguments}


\section{Zus�tzliche Features}

Die Mextools-Dateinen stellen eine einfache, lightweight Matrix-Klasse sowie
eine einfache Vektor-Klasse zur Verf�gung (siehe File mextools.h). 

\begin{itemize}
\item class mmatrix
\item class mvector
\end{itemize}
Diese k�nnen sowohl auf bestehende mxArrays aufgesetzt werden, wobei sie auf
den Datenbereich des mxArrays zugreift, als auch selbst Speicher allokieren,
der am Ende der Lebensdauer des Objektes automatisch wieder freigegeben wird.
Diese Klasse dienen damit haupts�chlich dazu, dem Programmierer bequemen Zugriff
auf die Elemente einer reellen Matrix bzw. eines reellen Vektors zu geben (eins-basierte
Notation).

\end{document}
